%%
%% This is file `sample-sigchi.tex',
%% generated with the docstrip utility.
%%
%% The original source files were:
%%
%% samples.dtx  (with options: `sigchi')
%% 
%% IMPORTANT NOTICE:
%% 
%% For the copyright see the source file.
%% 
%% Any modified versions of this file must be renamed
%% with new filenames distinct from sample-sigchi.tex.
%% 
%% For distribution of the original source see the terms
%% for copying and modification in the file samples.dtx.
%% 
%% This generated file may be distributed as long as the
%% original source files, as listed above, are part of the
%% same distribution. (The sources need not necessarily be
%% in the same archive or directory.)
%%
%% The first command in your LaTeX source must be the \documentclass command.
\documentclass[sigchi]{acmart}

%%
%% \BibTeX command to typeset BibTeX logo in the docs
\AtBeginDocument{%
  \providecommand\BibTeX{{%
    \normalfont B\kern-0.5em{\scshape i\kern-0.25em b}\kern-0.8em\TeX}}}

%% Rights management information.  This information is sent to you
%% when you complete the rights form.  These commands have SAMPLE
%% values in them; it is your responsibility as an author to replace
%% the commands and values with those provided to you when you
%% complete the rights form.
\setcopyright{acmcopyright}
\copyrightyear{2018}
\acmYear{2018}
\acmDOI{10.1145/1122445.1122456}

%% These commands are for a PROCEEDINGS abstract or paper.
\acmConference[Woodstock '18]{Woodstock '18: ACM Symposium on Neural
  Gaze Detection}{June 03--05, 2018}{Woodstock, NY}
\acmBooktitle{Woodstock '18: ACM Symposium on Neural Gaze Detection,
  June 03--05, 2018, Woodstock, NY}
\acmPrice{15.00}
\acmISBN{978-1-4503-XXXX-X/18/06}


%%
%% Submission ID.
%% Use this when submitting an article to a sponsored event. You'll
%% receive a unique submission ID from the organizers
%% of the event, and this ID should be used as the parameter to this command.
%%\acmSubmissionID{123-A56-BU3}

%%
%% The majority of ACM publications use numbered citations and
%% references.  The command \citestyle{authoryear} switches to the
%% "author year" style.
%%
%% If you are preparing content for an event
%% sponsored by ACM SIGGRAPH, you must use the "author year" style of
%% citations and references.
%% Uncommenting
%% the next command will enable that style.
%%\citestyle{acmauthoryear}

%%
%% end of the preamble, start of the body of the document source.
\begin{document}

%%
%% The "title" command has an optional parameter,
%% allowing the author to define a "short title" to be used in page headers.
\title{CSCM37: HCI Gamification in Education Report}

%%
%% The "author" command and its associated commands are used to define
%% the authors and their affiliations.
%% Of note is the shared affiliation of the first two authors, and the
%% "authornote" and "authornotemark" commands
%% used to denote shared contribution to the research.
\author{Andrew Gray}
%\authornote{Both authors contributed equally to this research.}
\email{445348}
\orcid{1234-5678-9012}
\affiliation{%
  \institution{Swansea University}
}


%%
%% By default, the full list of authors will be used in the page
%% headers. Often, this list is too long, and will overlap
%% other information printed in the page headers. This command allows
%% the author to define a more concise list
%% of authors' names for this purpose.
\renewcommand{\shortauthors}{Trovato and Tobin, et al.}

%%
%% The abstract is a short summary of the work to be presented in the
%% article.
\begin{abstract}
  Within the report, we will be looking into gamification, what it is and how itis implemented within education. Gamification is the technique of using game-like features, or a game layer, on non-traditional game methods. In regards to education,  gamification is one of two camps. One camp focuses on gamification layer that is on top of a traditional teaching method, but that do not include games.  On the other hand, in the other camp, they acknowledge that games can be classes as gamification, within education, as long as the focus is about learning. From the author's previous experience, an acknowledgement, of working within a school, gamification elements have been used daily within education. However, it has not been portrayed as such, on a day to day basis. With the core principles of what the gamification technique is trying to do, it gotten lost within the everyday mill of school life. It has, therefore, just been perceived as just another educational tool. A solution to this problem has been proposed. The aims are to add these key school targets, of the students, be presented to the students within a manner that they are more accustomed to, which is in the form of a mobile application. With a detailed plan of how this development of the app will be carried out. A proposed research plan on how the research will be carried out, on the app, to gauge its effectiveness within its representatives. 
\end{abstract}

%%
%% The code below is generated by the tool at http://dl.acm.org/ccs.cfm.
%% Please copy and paste the code instead of the example below.
%%
\begin{CCSXML}
<ccs2012>
 <concept>
  <concept_id>10010520.10010553.10010562</concept_id>
  <concept_desc>Computer systems organization~Embedded systems</concept_desc>
  <concept_significance>500</concept_significance>
 </concept>
 <concept>
  <concept_id>10010520.10010575.10010755</concept_id>
  <concept_desc>Computer systems organization~Redundancy</concept_desc>
  <concept_significance>300</concept_significance>
 </concept>
 <concept>
  <concept_id>10010520.10010553.10010554</concept_id>
  <concept_desc>Computer systems organization~Robotics</concept_desc>
  <concept_significance>100</concept_significance>
 </concept>
 <concept>
  <concept_id>10003033.10003083.10003095</concept_id>
  <concept_desc>Networks~Network reliability</concept_desc>
  <concept_significance>100</concept_significance>
 </concept>
</ccs2012>
\end{CCSXML}

\ccsdesc[500]{Computer systems organization~Embedded systems}
\ccsdesc[300]{Computer systems organization~Redundancy}
\ccsdesc{Computer systems organization~Robotics}
\ccsdesc[100]{Networks~Network reliability}

%%
%% Keywords. The author(s) should pick words that accurately describe
%% the work being presented. Separate the keywords with commas.
\keywords{HCI, Gamification, Gamification in Education, CHI, GRN}


%%
%% This command processes the author and affiliation and title
%% information and builds the first part of the formatted document.
\maketitle

\section{Introduction}
Presented to us has been a task to research within a subtopic of Human-Computer Interaction (HCI). The subtopic we chose was gamification and gamification within education. Researching into gamification within educations has been influenced by the author's previous experience of being a teacher and working within schools. We wanted to find out what the context of gamification is, and how it can be used within education, to take aspects of teaching and learning that can be brought into the 21st Century. To make aspects of education more accessible to students in a manner that they are more accustomed to in their everyday lives. 


Gamification, a term first coined in 2002, is an HCI technique used to add a game layer to traditional non-game like situations. The gamification aims to create extrinsic motivators for a person to be encouraged to do particular actions. Each action, upon completion, will have a little reward which, upon doing so, will release dopamine into the brain. The release of dopamine creates a good feeling within the participant's brain, which in turn is encouraging them to do it again. These rewards can be in the form of badges, achievements or progress bars, to name a few.

We have proposed a solution that will aim to provide all teachers and students with an application to streamline all the key metrics that students need to know and have ownership of. These metrics involve attendance, attainment progress, for example, test results, class and homework scores. These metrics have forms of rewards attached to them, which are recorded through the year with either end of year or end of term awards. However, these metrics, in order for the students to know them, rely on the teachers, providing them with it. In most cases their form tutor, but these do not always get passed on. This is usually because of the sheer amount that needs doing in the short amount of time available. 


The report will first look at the history of gamification, where it started and what evolution it made over time to where it is current stance is. We will then explain the science behind gamification and why it works, looking at why it is a powerful tool and how gamification is viewed and used within education. We will then explain a problem, that has been observed within education, along with a possible solution to this problem. We will finally look at the solution and a proposed plan of how this solution will be created and researched to see how effective it could be within the intended environment. To finish a summary with areas of potential future work that could be done with the solution.

\subsection{History}

The term gamification first appeared in the context of software design in 2008 \cite{4}, but the term only started to get more widespread recognition within 2010. However, the term "gamification" was first coined by Nick Pelling in 2002 \cite{3e}. Its initial aim was to incorporate the social and reward features of games into the software. Gamification started to gain much attention, so much so that it got described by a venture capitalist as one of the most promising areas of gaming \cite{5}. Gamification is now known as a powerful tool for engagement, which has, since its initial conception, now become a standard feature within software development \cite{3e}. Researchers consider gamification to be the progression of earlier work that focuses on adopting game-design elements to non-game situations and contexts. Research in the HCI field, in regards to apps that use game-driven elements for motivation and also in interface design, suggest that there is a connection between Soviet concepts of socialist competition and the American management trend of "fun at work" \cite{5}.

Jane McGonigal, in 2010, delivered a groundbreaking TED Talk titled, "Gaming Can Make a Better World' \cite{6}. This talk gets reflected as the defining moment in the history of gamification. Within the talk, she foretells a game based utopia. Where she states that "When I look forward to the next decade, I know two things for sure: that we can make any future we can imagine, and we can play any games we want, so I say: Let the world-changing games begin \cite{6}." Hindsight tells she was correct, as, from 2011, gamification starts to pick up steam.  At a Computer-Human Interaction (CHI) conference, a workshop titled "Gamification: Using Game Design Elements in Non-Gaming Contexts \cite{7}", which generated the Gamification Research Network (GRN) \cite{11}, in the year 2011. Through the years 2012 to 2016, gamification continues to grow. Even so, that gamification goes viral without people knowing through a game called Pokémon Go. Pokémon Go is one of the most successful applications of gamification with over 800 million downloads. People who would usually turn their nose up at badge collecting were out patrolling the streets searching for rare Pokémon. Pokémon Go is one of the most successful apps of all time. It even broke records \cite{3e,8}. It could be said thanks to Pokémon Go, that gamification is now everywhere. 

Many established technology and other companies, including SAP AG, Microsoft, IBM, SAP, LiveOps, Deloitte, and other companies have started using gamification in various applications and processes \cite{9}. 

The increased popularity in gamification, within some contexts, has had led to many legal restrictions be placed upon it, especially when linked to the Internet of Things (IoT) features. However, this mainly refers to the use of virtual currencies and assets, as well as data privacy, data protection and labour laws. These laws are due to its nature of being a data mining systems that spread information online, known as data aggregator \cite{10, 11}. 


%%%%%%%% This is where I am converting!!!%%%%%%%%%% 

\subsection{Science of Gamification}
%%%%%


Whether it is playing a more traditional video game, mobile game, or a recent phenomenon McDonald's Monopoly, we can all agree that games are fun, and there is no denying that.  The games industry is work an estimated \$2.3 trillion, showing that the global entertainment and media business is massive everywhere \cite{12}. There is a reason behind this, as games made are crafted with the human brain in mind. From each roll of the dice, getting the correct combination, to defeating an opponent and enemy, to building a new settlement, each action rewards the brain, and its reward centre lights up \cite{13}. 

By incorporating aspects from games like points, levels and progression bars into non-game situations, we can recreate the experience of gaming. Having these elements within a product, to interact with the user, is why gamification is so powerful. 

Games ranging from Super Mario Bros. to Monopoly have a real impact on brains and the way we learn. These impacts, on our brain, is due to dopamine. Dopamine is a neurotransmitter that is triggered within a person whenever they do something positive or when a person feels that they have achieved something \cite{14}. In essence, dopamine is a natural drug that makes people feel good \cite{13}. This drug, dopamine, is an integral part of our learning through reinforcement learning. As Nestler Lab explains, "activation of the pathway tells the individual to repeat what it just did to get that reward \cite{13, 15}." We do something well, and we get a sense of reward from our brains which leads us to do it again. Hence why we as humans tend to feel good when we are learning something; however, it is not very easy to stay motivated while learning as the learning requirements increases. At this stage is where gamification shines and can help keep the user/learner motivated with a little boost along the way. The motivation, the critical factor gamification tries to manipulate, is triggered by the sense of success. Which leads onto more willingness and desire to do something, this can be achieved by not only rewarding the final goal but by also releasing small amounts of dopamine as we are edging closer to a goal. Allowing a user to know if they are nearing a milestone can be achieved by using progress bars. Each sub-goal completed fills up the bar giving instant gratification, with small hits of satisfaction and dopamine, on the build-up to meeting the primary goal and that massive hit of dopamine, therefore creating that motivation to keep going. This situation becomes superseded only when an unexpected gratification situation occurs, releasing even more dopamine. 

While motivation is at the centre of gamification, our enthusiasm comes from three main areas: Autonomy; Value; Competence \cite{16}. When someone is in charge of their destiny, they are more motivated to succeed. Empowering the person with more control will mean that they will work harder towards completing objectives, especially for a more extended period, which is even more true when the person gets given the opportunity and authority to select their direction when solving a problem. This aspect gets regarded as giving the person autonomy. The second principal area, value, is about the person feeling a sense of value to an activity or action that they are doing. If the person feels that there is self-worth to the activity, then they will increase interest in the activity and increase their motivation levels. Research states that a positive correlation occurs when a student values a subject at school and their willingness to investigate a question. If the person cares, they will keep going and work harder until the task gets completed \cite{12,16}. Finally, the third area, of importance, is competence. If a person develops a certain degree of proficiency at something, they are more likely to keep doing it. Another study has shown that there is a link between a student's sense of mastery and their desire to continue certain activities. Those who give credit to natural talent rather than hard work will more likely give up more quickly. 

Gamification aims to take advantage of our extrinsic motivations, factors like final grades or money, and intrinsic motivation, traits like personal gains or enjoyment, to try and enhance our daily activities or tasks. In order for the gamification to be most effective, then both these motivation factors need to be accounted for, within the task. In order for the person to feel good about oneself, a form of reward has to exist \cite{12}. 

\subsection{Gamification in Education}

The gamification of learning is an educational approach to motivate students to learn by using game elements in a learning environment \cite{22}. Gamification in education is very much the same thing as gamification in general, but with more of a focus on learning. However, gamification in learning has two main views within HCI academia. One side categories gamification of learning as learning that has game-like features, but only when the learning is happening in a non-game context, like a classroom. This version would involve a range of components that get presented in a system, or game layer, which aims to happen alongside the learning in a conventional classroom. At the same time, the other half includes games that have been designed to induced learning within them \cite{22}. 

Gamification, within an educational or a learning situation, has multiple advantages. It is not just about trying to improve attendance with incentives by reaching a particular score, or extra rewards for completing specific tasks within a lesson. It can aid in cognitive development in adolescents, which can increase levels of engagement and can aid with accessibility within the classroom \cite{24}. Games that get produced for enhancing cognitive development are known as "rain games" \cite{24}. These popular games typically are focused around a series of questions and problems for the player to solve or answer. These games develop the rate the player can sustain information and increase the brain's ability to process information. The levels of the engagement of the students' increases, when gamification has been used, within a classroom. A study performed by scientists aimed to measure the students' levels of engagement in a classroom where gamification elements are applied \cite{25}. They assigned a point system to multiple daily activities. Every student had a measurement of the perceived level of engagement. The finding showed that the game like setting was supporting the learning within the classroom and increased productivity. Therefore, by increasing engagement levels, it also means it helps students be able to access the content of the lesson, that is or needs to be delivered better. 

Even though gamification can aid teaching students of all needs, a study conducted on students who had autism using video games showed that this training package was powerful in teaching content that was age-appropriate \cite{26}. However, gamification of learning is not something just for the classroom; its an excellent tool for learning outside the classroom. Games like Spore create a deeper understanding of life and evolution as the game simulates a world where the player's character will evolve, adapting to their surroundings through reproduction. Another game by the same creator, Will Wright, Sim City aims to teach the player key skills like \cite{27}: Supply and demand; Budgeting; Urban planning; Managing the environment; Understanding utilities and services like transport systems and public services; Reading and maths skills. 

Gamification of learning has excellent potential benefits. The benefits involve \cite{22}: Allowing students to have ownership of their learning, as well as giving opportunities for the learner to gain a sense of their' own identity" through alternative role-playing selves. The freedom, without any negative repercussions, to fail and keep on trying again. The ability to increase fun and joy while learning. The opportunity for tasks to be differentiated. Making the learning visible and providing opportunities to inspire intrinsic motivators for learning. Also, the ability to aiding in motivating students with low levels of motivation. 

\section{Challenges}

With previously being a teacher, it got observed that many gamification aspects are implemented within day-to-day aspects of schools already. Schools have metrics that staff and students must know and monitor their attendance, progress, behaviour and extra curriculum achievements. While teachers are usually completing paperwork to evidence that they are tracking these metrics, in forms of reports and spreadsheets, students are reliant on their form teacher or subject teacher to pass on the information. However, most of the general school metrics, like attendance and overall performance of the student gets left to the form teacher to pass onto the students. Usually, these metrics have an end of term or year reward, end of year trip or in the case of year 11 students, a passport to the prom. It is only allowing them to be allowed to attend if their metrics are at a certain level. Unfortunately, this form time tends to be only for a brief period of the day, most cases being about twenty minutes; within most cases, a lot of tasks and information needs to be passed out. For example, within the twenty minutes, teachers are expected to do the register, pass on any whole school messages, complete some form activity, which is usually, a citizenship topic or life skill theme, as well as any other task like panner signing. As we can see, a lot to do in a short period. Therefore, pass on this vital information gets lost and creates a lack of ownership from the students of their performance metrics. Due to the lack of ownership by the students, excluding a minority of extremely engaged students, most students rarely take an interest in their overall picture. Therefore their lack of understanding into the effects these particular metrics has on their education. Of their attendance and attitude, based on their behaviour and academic performance has on their more significant experience and impact on their education. Schools usually do have award systems in place and attendance tracking in place, but it is all done by separate pieces of software. With still a massive reliance on the teachers pass on the information or logging into the systems that distribute the behaviour points. Due to this lack of consistency from the teachers and inability for students to gain the information that they desire in a manner that is more suitable to their everyday life, the gamification techniques that are in place get lost and lose their impact and meaning. 

\section{Possible Solution}

A possible solution is to be able to create a platform that makes everything centralised. Education institutes already use a register system called SIMS[reference sims]. We believe this could be the center point for everything, in terms of input of data from the teachers. By enabling all of this information to be inputted in one place, like the praise points, behaviour points, a log of homework that has been set received on time or late, or even the passport to prom, it allows all of this information to be extracted more streamlined. With this one point of entry of the data, it should enable most teachers to be consistent and be more onboard to carry out the required base tasks that are required for the main school metrics that the students need to know.

Form having the initial base information, like the attendance and behaviour points, a real-time app could extract this information and start to gamify it in a manner that today’s youth would be more accustomed. Gamification features like a progress bar filling up based on the number of days a student attends. Ultimately working towards that end of year trip, or maybe the passport to prom.

Help to motivate students. A daily personalised notification could be out to students. Maybe reminding them to bring their PE kit, or that they have a particular class after school. It could also offer words of encouragement with students who maybe have low attendance by giving fact about what a lack of attendance can do to their education. Alternatively, encouraging information about having a better outlook, maybe motivational quotes on improving behaviour. The motivation for the students could also be driven with badges. These badges being awarded for the students having a ‘hot streak’ of attendance, like Snap Chat’s streak of messaging, or badges awarded for homework completed on time.

The key focus for the students would have an app that they could view on a computer or their smartphones, which provides them with all the required metrics that they need to know about. This app, as a result, allows the students to have autonomy over their metrics and be able to view them in the best manner for them. Not having to rely on a teacher passing out the information, and fundamentally in a format that they are used to. However, we do understand that a lot of schools restrict mobile phones, or even ban them out right. But, with changing times, with schools wanting students to be more mindful and resilient, we need to make sure that they can get the vital information that is required, fundamentally in a way that they are more likely to receive it. We do believe that schools views on mobile phones should be taken into account. However, it is also naive to not take into account the changing worlds landscape and the type of world these current students are growing up in. Forever changing, so doing something, in a way, because that is how it was done while we were in school, or because it is the way it has always been done, does not make it right. Not in the slightest. We need to be training our children skills that are needed for a world that has not been created yet, using tools that are new and innovative. Not with tools that are traditional or within the teachers comfort zone.

\subsection{Quantitive  or Qualitative Research}
Quantitive research is a method of research that involves collecting numerical data, which we then analyse by using mathematically based methods which is usually within the statistics field of mathematics \cite{quantitivelectures}. While qualitative research methods are about discovering why and how people behave the way they do. In order to provide in-depth information about human behaviour in regards to observations, field studies, focus groups and interviews \cite{quantitivelectures}. 

Due to the nature of the problem, we believe a quantitative research method would be the best method to use in regards to our possible solution. Due to the nature of the problem, we want to gain information that will inform us of how well the students and teachers adopt the application. In addition to the levels of engagement from both students and teachers. Especially in regards to students and their ownership of their metrics. To see if the engagement level has increased while using the app, as to the traditional methods. Instead of finding out how and why people have been using the application.

As we would be following a quantitive research method, there are nine steps that we will need to follow when carrying out our research. These nine stages of the procedure are:

\begin{enumerate}
	
	\item Identify a hypothesis and the required measures needed
	\item Specify a study design
	\item Build the assessment materials
	\item Submit the study designs for ethics approval
	\item Run the initial pilot
	\item Recruit participants
	\item Run the study
	\item Analyse the data
	\item Write the findings paper, which acknowledges any of the limitations of the work
\end{enumerate}

While we are not conducting out this research in this report, we will, however, layout the stages that we would carry out, stating what we would do in each and why this would be the case. 

\subsection{Hypothesis}

At this stage, we will have to at first develop an aim or a hypothesis from potential ideas and any reviews. In regards to our potential solution to the problem that we have identified, the main aim is about increasing the student's ownership of their metrics. That schools and students are measured against, which schools provide incentives for the students to improve them. They are ultimately aiming to motivate them to do better and increase their engagement within their education and their key metrics. While also trying to create a streamlined system tool, for the teachers marking and assessment, on students progression.

An example hypothesis we would use for the students is:\\\\
\textbf{H0}: There is no difference in the user uptake of the ownership of student metrics from using the app and traditional methods.\\
\textbf{H1}: There is a difference in the user uptake of the ownership of student metrics from using the app and traditional methods.\\\\
\textbf{Independent variables}: Student engagement with the app, student improvement of attainment of their metrics, and how to improve.\\
\textbf{Dependent Variables}: The detail of the information provided to students, ease of use of the app, encouraging methods to incentivise return to the application.\\

While an  example hypothesis we would use for the teachers is:\\\\
\textbf{H0}: There is no difference in the user experience in inputting student metrics compared to traditional methods.\\
\textbf{H1}: There is a difference in the user experience in inputting student metrics compared to traditional methods.\\\\
\textbf{Independent variables}: Teacher engagement with the app, Teacher improvement workflow of assessment tracking.\\
\textbf{Dependent Variables}: Time spent inputting data.\\

Once an initial app design has been developed, the study will need to have two main areas of focus. The first main area is the students' engagement and the effects it has had on their ownership of their metrics. For example, has the gamification techniques improved students' metric or their understanding of where they are. As well as how well the application streamlines the recording of the students' progress for the teachers. Is it able to complete all the required tasks to display all the information for the students without the teachers having to have too much input or repeating tasks that other systems they use do?

\subsection{Study Design}
As we have more than one independent variable, we would have to take a factorial design. We would use a split-plot design, which is due to the nature of students currently having a system in place at the moment, which is non-digitised. Therefore, we would aim to assess their current engagement with the system, then take selected classes from different years across the school. By doing this, this will enable us to see how much engagement has increased for the students using the app. We could also see if the students are not using the app within the experiment have a greater desire to use it. Another way we could test the effectiveness of the application is to have one whole school do it and another local school, due to having a similar area background for the participants, not use the app and then assess the levels of engagement.


\subsection{Usability Testing, Data Collection and Surveys}
Shneiderman presents eight fundamental rules to interface design \cite{shneiderman2016designing}. These golden rules are: Strive for consistency; Cater to universal usability; Offer informative feedback; Design dialogues to yield closure; Prevent errors; Permit easy reversal of actions; Support internal locus of control; Reduce short-term memory load. To make sure we are, as designers and developers, sticking to these golden rules, we will use usability testing for our first app.

Usability testing is gaining knowledge to the extent to which a product can be used by specified users to achieve specified goals with effectiveness, efficiency and satisfaction in a specified context use \cite{bevan2001proposed}. The main goal is to make sure we are identifying and fixing any user interface related issues or flaws as soon as possible. Interface related issues a categorised as interface components that impact on the performance of the app and creates any potential frustration. It is, however, not about general style preferences we are researching the interface, not the user. Researching the user is the overall  \cite{HCIusability}.  

Usability testing is a crucial stage for us within the development of the application. We believe that before we can start research, on the impact of the app on the learning environment, we must make sure the app is at a high level first, in terms of usability \cite{lazar2017research}. Having a good app design will allow us to research on how the app's functionality influences the way tasks are carried out and done as this should reduce any edge cases about the design of the app on the critical research, its impact within the school. 

When doing usability testing, we must use representatives that are a true representative of the intended users of the app. In our case, this would be students between the ages of 12 to 18 and adults who are teachers. 

In order to gain any insights into what our users think of the app, We will need to gather data on the user's usage and views of the app. The information gathered can give us valuable insights into the manors they use the app, and any heuristics approach they have, allowing us to tap into them. Two good ways to gain insights into the user's actions is to use web usability \& design and stored application data \cite{datacollectionlecture}. These methods will give us insights into how they are using the app, consciously and subconsciously, which will be giving us insights that they might not necessarily say in a survey.

Another way to get the views of the user is to use surveys. A survey is a set of well-written questions which are asked to an individual to respond \cite{surveylecture}. These are one of the most commonly used methods of research \cite{lazar2017research}. However, even though they are widespread, they do possess many benefits and some drawbacks. The benefits allow for the collection of large amounts of data from a large geographical area, while not needing any special equipment. However, they can lead to the data gets referred to as shallow as it is not getting in-depth answers within the data \cite{surveylecture}, and it can also lead to gaining biased data, especially when the questions get linked to patterns of usage \cite{lazar2017research}.

So overall, a combination of all the data collection methods will allow us to get an excellent in-depth overall view of what people think of the application.

\section{Conclusion and Future Discussions}
Nick Pelling created gamification in 2002. He intended to incorporate the social and rewarding features of games into the software. However, it was not until Jane McGonigal, in 2010, through her TED talk that gamification started to get much attention. The talk spawned the GRN the following year. A research group dedicated to the development of the standards of gamification. At this point, gamification truly started to be at the forefront of peoples minds. Although, it was not until the release of a mobile app game, called Pokémon Go in 2015, did gamification hit the mainstream. With nearly every kind of app using some form of gamification technique, like progress bars and awards, to name a few.

With regards to gamification in education, the scientific research community are split into two camps.  One camp that views gamification as adding game-like layers on top of more traditional settings, like a classroom and the other half also including computer games that have been developed with education and learning at the forefront. None the less, when gamification has been proven to have positive effects on students when used within the classroom, giving them a greater sense of ownership and autonomy levels.

Although educational establishments already have gamification features implemented. The features, which are usually based around key metrics that students are expected to know, that usually have incentives attached to them for the end of the year or term awards. However, these techniques are perceived to have been lost within day-to-day aspects of the school. 

We have proposed a solution to the current method that schools implement gamification in regards to the vital student metrics. The solution is an app called Stu Le'Dash. The app's aims to bring the metrics, that the students need to know, to them in a manner that they are more accustomed to, this being an app using gamification. With additional benefits for teachers of making their admin tasks more streamlined.

We would aim to create the app by using usability testing. Once the app has been created, we would use quantitive research methods to gain insights into how well the app increased the levels of ownership and interest in the students and the interest in their metrics. As well as looking into how much the teachers felt the app streamlined their tracking of the students.

Although having an app, to provide students with their key metrics, might seem to be going against many school policies of banning mobile phones. We believe the benefits it could bring would genuinely revolutionise the educational space. The app has many many potential areas where it could expand and develop. These areas could include sports fixtures and results for the school teams by using features that the IoT could provide and facilitate these.  

Another area where we believe that the next step for this app could be would be to introduce additional features like Virtual Reality/Augmented Reality (VR/AR) as well as assignment hand-in date calendar. The VR/AR could be a tool to help facilitate the students learning outside of school, while the calendar will help remind the students with the key dates as even though we are aiming for the students to take ownership of their school career, everyone needs a little reminder now and then. The overall aim is to aim to make everything as streamlined as possible, all in one place, for both students and teachers.
%%
%% The acknowledgments section is defined using the "acks" environment
%% (and NOT an unnumbered section). This ensures the proper
%% identification of the section in the article metadata, and the
%% consistent spelling of the heading.
\begin{acks}
We would like to thank Dr. Siyuan Liu for providing delivering lectures and providing the lecture slides.
\end{acks}

%%
%% The next two lines define the bibliography style to be used, and
%% the bibliography file.
\bibliographystyle{ACM-Reference-Format}
\bibliography{sample-base}



\end{document}
\endinput
%%
%% End of file `sample-sigchi.tex'.
